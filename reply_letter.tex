% Template for replying to reviewer comments
%
% Author: Gertjan van den Burg (https://gertjan.dev)
% Source: https://github.com/GjjvdBurg/LaTeXReviewReplyTemplate
% License: MIT
% 
% Command usage:
%
%  - Reply: complete answer to the point raised.
%
%  - Partial reply: partial answer to the point raised. More text may be 
%    needed. Can also imply work on the paper that still has to be done.
%
%  - Todo reply: Used for discussion and/or sketching a reply.
%
%  - Need reply: Signals that a point needs to get a reply.
%
\documentclass[11pt,a4paper]{article}

\usepackage{geometry}
\usepackage[most]{tcolorbox}

% Colors
\definecolor{lightgray}{rgb}{0.9, 0.9, 0.9}
\definecolor{lightyellow}{rgb}{0.98,0.91,0.71}
\definecolor{lightred}{rgb}{0.8,0.4,0.4}

% Boxes
\newtcolorbox{replybox}{colback=lightgray, grow to right by=-10mm, grow to 
	left by=-10mm, boxrule=0pt, boxsep=0pt, breakable, before skip=10pt}
\newtcolorbox{partialbox}{colback=lightyellow, grow to right by=-10mm, grow to 
	left by=-10mm, boxrule=0pt, boxsep=0pt, breakable, before skip=10pt}
\newtcolorbox{todobox}{colback=lightred, grow to right by=-10mm, grow to left 
	by=-10mm, boxrule=0pt, boxsep=0pt, breakable, before skip=10pt}

% Counters
\newcounter{epntcnt}
\newcounter{rpntcnt}
\newcounter{revcnt}

% Environments
\newenvironment{reviewer}{%
	\refstepcounter{revcnt}%
	\setcounter{rpntcnt}{0}%
	\section*{Reviewer \therevcnt}%
}{}
% Editor environment takes an optional argument
\newenvironment{editor}[1][Editorial Decision]{%
	\setcounter{epntcnt}{0}%
	\section*{#1}%
}{}

% Reply Commands
\newcommand{\reply}[1]{%
	\begin{replybox}%
		#1
	\end{replybox}%
}
\newcommand{\partialreply}[1]{%
	\begin{partialbox}%
		Partial reply: \emph{#1}
	\end{partialbox}%
}
\newcommand{\todoreply}[1]{%
	\begin{todobox}%
		Todo reply: \emph{#1}
	\end{todobox}%
}
\newcommand{\needsreply}{%
	\todoreply{This comment needs a reply.}%
}

% Point command differs depending on the environment
\makeatletter
\newcommand{\point}{%
	\medskip\noindent%
	\def\@tmp{editor}%
	\ifx\@tmp\@currenvir \refstepcounter{epntcnt}\theepntcnt\textbf{.}  
	\else%
	\def\@tmp{reviewer}%
	\ifx\@tmp\@currenvir \refstepcounter{rpntcnt}\therpntcnt\textbf{.}  
	\else\fi\fi%
}
\makeatother

% Renew the commands for the point counters to get the formatting we want in 
% both the references and the item labels.
\renewcommand{\therpntcnt}{\textbf{\textup{\therevcnt.\arabic{rpntcnt}}}}
\renewcommand{\theepntcnt}{\textbf{\textup{E.\arabic{epntcnt}}}}

\setlength\parindent{0pt}
\newcommand{\bft}[1]{\mathbf{#1}}

\title{Reply to Reviewer Comments}
\author{John Doe}

\begin{document}
Lorem ipsum dolor sit amet, consectetur adipiscing elit, sed do eiusmod tempor 
incididunt ut labore et dolore magna aliqua. Ut enim ad minim veniam, quis 
nostrud exercitation ullamco laboris nisi ut aliquip ex ea commodo consequat. 
Duis aute irure dolor in reprehenderit in voluptate velit esse cillum dolore 
eu fugiat nulla pariatur. Excepteur sint occaecat cupidatat non proident, sunt 
in culpa qui officia deserunt mollit anim id est laborum.\\

The Authors\\

\today.

\begin{editor}
	% NOTE: Here you would put the editor comment verbatim after each 
	% point that needs addressing, you can place your reply. The 
	% \needsreply command is used to mark points that need reply while the 
	% revision is in process
	Some initial comments.

	\point Wow what a nice paper
	\reply{Thank you}

	\point Please defend your arguments better.
	\reply{\label{ed:defend} Done}

	Other comment that doesn't need a reply.
\end{editor}

\begin{reviewer}
	% Again, use \point and \reply where appropriate. Points can be 
	% labelled and referenced using \label and \ref as usual.
	This is a very interesting paper.

	\point Arguments need to be defended better.
	\reply{Fixed, see also our reply to comment \ref{ed:defend}.}

	\point There's a typo on page 10.
	\needsreply
\end{reviewer}

\end{document}
